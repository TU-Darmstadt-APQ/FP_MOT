\documentclass[9pt,a4paper]{extarticle}
\usepackage{geometry}
\usepackage{amsmath}
\usepackage{amsfonts}
\usepackage{amssymb}

% Adjust page margins
\geometry{
textwidth = 15.6cm,
textheight = 24.7cm,
}
\pagestyle{empty}
\begin{document}
\subsection*{Eindimensionale Betrachtung}
Annahme: Gaußverteilung zu $t = 0$
\begin{align}
n(r, t= 0) \varpropto \exp\left(\frac{-x^2}{2 \sigma_x^2}\right)
\end{align}
Mit der eindimensionalen Geschwindigkeitsverteilung (Maxwell-Boltzmann) 
\begin{align}
g(v,T) = \left[\frac{m}{2 \pi k T}\right]^{\frac{1}{2}} \exp \left(\frac{-mv^2}{2kT}\right)
\end{align}
gilt dann zum Zeitpunkt $t>0$:
\begin{align}
n(r, t> 0) &\varpropto \int \mathrm{d}v \exp\left(\frac{-(x-vt)^2}{2 \sigma_x^2}\right) \cdot g(v,T) \\
&\varpropto \int \mathrm{d}v \exp\left(\frac{-(x-vt)^2}{2 \sigma_x^2}\right) \cdot \exp \left(\frac{-mv^2}{2kT}\right) \\
&= \int \mathrm{d}v \exp\left(\frac{-x^2}{2 \sigma_x^2} + \frac{2xvt}{2 \sigma_x^2} - \frac{v^2 t^2}{2 \sigma_x^2} - \frac{mv^2}{2kT}\right) \\
&= \int \mathrm{d}v \exp\biggl(\frac{-x^2}{2 \sigma_x^2} + v \underbrace{\left( \frac{xt}{\sigma_x^2}\right)}_{b} - v^2 \underbrace{\left(\frac{t^2}{2 \sigma_x^2} + \frac{m}{2kT} \right)}_{a^2}\biggr) 
\end{align}
Bronstein, Wolfram o.ä. liefert:
\begin{align}
n(r, t> 0) &\varpropto \exp\left(\frac{-x^2}{2 \sigma_x^2}\right) \cdot \exp \left(\frac{b^2}{4a^2}\right) \\
&= \exp\left(\frac{-x^2}{2 \sigma_x^2}\right) \cdot \exp \left(\frac{\left( \frac{xt}{\sigma_x^2}\right)^2}{4\left(\frac{t^2}{2 \sigma_x^2} + \frac{m}{2kT} \right)}\right) \\
&= \exp\left(\frac{-x^2}{2 \sigma_x^2} + \frac{x^2t^2}{4\sigma_x^4\left(\frac{t^2}{2 \sigma_x^2} + \frac{m}{2kT} \right)}\right) \\
&= \exp\left(\frac{-x^2}{2 \sigma_x^2} + \frac{x^2t^2}{2\sigma_x^4\left(\frac{t^2}{\sigma_x^2} + \frac{m}{kT} \right)}\right) \\
&= \exp\left(\frac{-x^2 \cdot \sigma_x^2\left(\frac{t^2}{\sigma_x^2} + \frac{m}{kT} \right) +x^2t^2}{2\sigma_x^4\left(\frac{t^2}{\sigma_x^2} + \frac{m}{kT} \right)}\right) \\
&= \exp\left(\frac{-x^2 t^2 -  x^2 \sigma_x^2 \frac{m}{kT} +x^2t^2}{2\sigma_x^4\left(\frac{t^2}{\sigma_x^2} + \frac{m}{kT} \right)}\right) \\
&= \exp\left(\frac{- x^2 \sigma_x^2 \frac{m}{kT}}{2\sigma_x^4\left(\frac{t^2}{\sigma_x^2} + \frac{m}{kT} \right)}\right) \\
&= \exp\left(\frac{- x^2 \frac{m}{kT}}{2\sigma_x^2\left(\frac{t^2}{\sigma_x^2} + \frac{m}{kT} \right)}\right) \\
&= \exp\left(\frac{- x^2 \frac{m}{kT}}{2t^2 +2 \sigma_x^2 \frac{m}{kT} }\right) \\
&= \exp\left(\frac{- x^2 }{2\left(\frac{kT}{m}t^2 + \sigma_x^2\right)}\right) 
\end{align}
Die Atomdichte entspricht also wieder einer gaußschen Verteilung, allerdings mit breiteren Varianz
\begin{align}
\sigma_x^2 \left(t\right) = \frac{kT}{m}t^2 + \sigma_x^2
\end{align}
\end{document}